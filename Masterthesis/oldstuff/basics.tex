\chapter{Framework for analytical model}
\label{ch:basics}

\section{Theory of superconductivity}

The discovery of the isotope effect in 1950 revealed that not only lattice electrons but rather the whole lattice determines the superconducting properties of a solid. Experiments measuring the critical temperature $T_c$ of different mercury isotopes showed that indeed, there is a relation between the isotope mass and $T_c$. Herbert Fr\"olich was then the first to introduce a new concept to explain superconductivity. He showed that a phonon-intermediated interaction between electrons and the lattice could lead to an attractive long-range interaction of electrons in the lattice. Figuratively speaking, an electron passing through the crystal lattice will polarize it by attracting the positive ions. It leaves a deformed lattice, which will then attract a second electron. An effective attractive interaction between these two electrons is created. % – they form a Cooper pair. 
In 1956, Cooper showed that the electronic ground state, the Fermi sea at $T = 0$, is unstable if a weak attractive interaction is taken into account. This layed the foundation of the BCS theory \cite{Bardeen1957}, the first microscopic theory after the discovery in 1911 by Heike Kammerlingh Onnes. 

%TODO maybe include picture that explains, why opposite k vectors
%Cooper instability

\subsection*{Formulas needed}
Hamiltonian:
\begin{align}
H &= H_0 + H_1 \label{eq:H}\\
H_0 &= \sum_{\mathbf{k}, \sigma} \xi_{\mathbf{k}} c^{\dagger}_{\mathbf{k} \sigma }c_{\mathbf{k} \sigma }  \label{eq:H0}\\
H_1 &= \frac{1}{N} \sum_{\mathbf{k}, \mathbf{k'}} V_\mathbf{{\mathbf{k}, \mathbf{k'}}} c^{\dagger}_{\mathbf{k} \uparrow }c^{\dagger}_{- \mathbf{k} \downarrow}   c_{- \mathbf{k'} \downarrow} c_{\mathbf{k'} \uparrow} \label{eq:H1}
\end{align}
The operators $c^\dagger_{\mathbf{k}, \sigma} , c_{\mathbf{k}, \sigma}$ are fermion operators that create or annihilate an electron with momentum $\mathbf{k}$ and spin $\sigma$. The first term in the Hamiltonian $H$ is the unperturbed electron Hamiltonian $H_0$ with parabolic energy dispersion $\xi_{\mathbf{k}}$. The  second term is the interaction Hamiltonian $H_1$, expressing the scattering of two electrons from $(- \mathbf{k'} \downarrow,  \mathbf{k'} \uparrow)$ to $(\mathbf{k} \uparrow , - \mathbf{k} \downarrow)$. The interaction potential $V_\mathbf{{\mathbf{k}, \mathbf{k'}}}$ exchanges the scattering for electrons with energy $|\xi_\mathbf{k}| \lesssim \hbar \omega_D$.
%TODO check!
The Hamiltonian in eq. (\ref{eq:H}) can be simplified by doing a mean-field approximation. In this approximation, an operator $A$ is expressed by a sum of its statistical mean $\braket{A}$ and small statistical fluctuations $\delta A$. Since the fluctuations are assumed to be small, terms with $\mathcal{O}((\delta A)^2)$ can be neglected.
\begin{align}
A &= \langle A \rangle + \delta A, \quad B = \langle B \rangle + \delta B  \nonumber \\
A B &= \langle A \rangle  \langle B \rangle  + \langle A \rangle  \delta B +  \langle B \rangle \delta A + \underbrace{\delta A \delta B}_{\approx 0}\label{eq:meanfield-der}
\end{align}
Using $\delta A = A - \langle A \rangle$ and inserting this back into eq. (\ref{eq:meanfield-der}) leads to
\begin{equation}
AB = \langle A \rangle B + \langle B \rangle A - \langle A \rangle \langle B \rangle.\label{eq:meanfield-ab}
\end{equation}
This approximation is applied to the interaction part $H_1$ in eq. (\ref{eq:H1}), replacing
\begin{equation}
A = c^{\dagger}_{\mathbf{k} \uparrow }c^{\dagger}_{- \mathbf{k} \downarrow}, \quad B =  c_{- \mathbf{k'} \downarrow} c_{\mathbf{k'} \uparrow} .
\end{equation}
The result is the BCS-Hamiltonian
\begin{equation}
H_{\text{BCS}} = \sum_{\mathbf{k}, \sigma} \xi_{\mathbf{k}} c^{\dagger}_{\mathbf{k} \sigma }c_{\mathbf{k} \sigma }    -  \sum_{\mathbf{k}} \Delta_{\mathbf{k}}^* c_{-\mathbf{k}, \downarrow} c_{\mathbf{k}, \uparrow}  - \sum_{\mathbf{k}} \Delta_{\mathbf{k}}  c^{\dagger}_{\mathbf{k} \uparrow} c^{\dagger}_{- \mathbf{k} \downarrow} + \text{const.}\label{eq:H-BCS}
\end{equation}
where 
\begin{eqnarray}
\Delta_{\mathbf{k}} &:=& - \frac{1}{N} \sum_{\mathbf{k'}} V_\mathbf{{\mathbf{k} \mathbf{k'}}} \langle c_{-\mathbf{k'}, \downarrow} c_{\mathbf{k'} \uparrow} \rangle \\
\Delta_{\mathbf{k}}^* &:=& - \frac{1}{N} \sum_{\mathbf{k'}} V_\mathbf{{\mathbf{k}, \mathbf{k'}}}  \langle  c^{\dagger}_{\mathbf{k'} \uparrow} c^{\dagger}_{- \mathbf{k'} \downarrow} \rangle
\end{eqnarray}
%TODO why is this the pair potential?
The BCS-Hamiltonian in eq. (\ref{eq:H-BCS}) can be diagonalized using the Bogoliubov transformation. The aim is to express the Hamiltonian in the basis of new fermion operators. These new operators will describe quasiparticles, which are a linear combination of $c^\dagger_{\mathbf{k}, \sigma}$ and $c_{\mathbf{k}, \sigma}$.
\begin{equation}
\begin{pmatrix}
\gamma_{\mathbf{k} \uparrow} \\ \gamma^{\dagger}_{-\mathbf{k} \downarrow}  
\end{pmatrix} = \begin{pmatrix}
u^*_{\mathbf{k} } & -v_{\mathbf{k} } \\
v^*_{\mathbf{k} }  & u_{\mathbf{k} } 
\end{pmatrix} 
\begin{pmatrix}
c_{\mathbf{k} \uparrow} \\ c^{\dagger}_{-\mathbf{k} \downarrow}  
\end{pmatrix}
\end{equation}\label{eq:bogol-trans}
Evaluating the fermion anticommutation relation using the transformation above yields
\begin{equation}
\left\{ \gamma_{\mathbf{k} \uparrow}, \gamma^{\dagger}_{\mathbf{k} \uparrow}  \right\}  = \dots = |u_{\mathbf{k}}|^2 | + v_{\mathbf{k}}|^2 \stackrel{!}{=} 1
\end{equation}
and will lead to the inverse transformation of eq. (\ref{eq:bogol-trans}). Inserting the inverse transformation into the BCS-Hamiltonian in eq.(\ref{eq:H-BCS}) will give the coefficients $u_{\mathbf{k}}$, $v_{\mathbf{k}}$ from eq. (\ref{eq:bogol-trans}) and finally yield to the diagonalized form of the BCS-Hamiltonian
\begin{eqnarray}
H_{\text{BCS}} &=&  \sum_{ \mathbf{k} \sigma } E_{ \mathbf{k} } \gamma^{\dagger}_{\mathbf{k} \sigma } \gamma_{\mathbf{k} \sigma }\\
E_{\mathbf{k}} &:=&  \sqrt{\xi^2_{\mathbf{k}}  + |\Delta_{\mathbf{k}}|^2 }
\end{eqnarray}
%TODO below:
\textbf{interpretation as holes, etc \\
plots of particle and hole exitation (dispersion relation with gap)\\
metion fermi surface and so on} \\

\subsection*{Bogoliubov de Gennes Hamiltonian}
%Motivation for BdG: Describing inhomogneous systems, example Josephson junctions --> need for a microsopic theory for inhomogenous systems. 
%Idea: make BCS- mean field hamiltonian spatially dependent. 
The ansatz for the BCS ground state used by Bardeen, Cooper and Schrieffer is based on the concept of Cooper pairs. It is a direct consequence of the instability in the ground state through the attractive interaction. The BCS theory proposes a BCS ground state built on eigenstates of the single-particle Hamiltonian $H_0$ from eq. (\ref{eq:H0}), leading to a ground state that consists of a linear combination of pair states. %TODO check!
\begin{eqnarray}
\ket{\psi_\text{BCS}} &=& \prod_{ \mathbf{k} } (u_\mathbf{k} + v_\mathbf{k} c^{\dagger}_{ \mathbf{k} \uparrow } c^{\dagger}_{ - \mathbf{k} \downarrow }) \ket{\text{vac}} \\
H_\text{BCS} \ket{\psi_\text{BCS}} &=& E_\text{BCS} \ket{\psi_\text{BCS}}  
\end{eqnarray}
In most cases however, a more realistic set-up or inhomogeneous system cannot be described in terms of eigenfunctions of $H_0$. With a vector potential $\mathbf{A} \neq 0$, for example, time reversal symmetry is not given any more. %TODO Überleitung (“In the general case”)
The characteristic length scale is the superconducting coherence length $\xi_0$. If a system is varying slowly over a length scale $l \approx \xi_0$, a spatially dependent, more general Hamiltonian is needed. 
In order to find an adequate expression for such a spatially dependent Hamiltonian, the following spinor is introduced
\begin{equation}
\ket{\Psi_\mathbf{k}} = \begin{pmatrix}
| \Psi_\mathbf{k_1} \rangle \\ | \Psi_\mathbf{k_2} \rangle
\end{pmatrix} := \begin{pmatrix}
c^{\dagger}_{\mathbf{k}, \uparrow} \\ c_{- \mathbf{k}, \downarrow}
\end{pmatrix} \ket{\psi_\text{BCS}} \label{eq:spinor}
\end{equation}
In this basis $\left\{| \Psi_\mathbf{k_1} \rangle, | \Psi_\mathbf{k_2} \rangle \right\}$, the Hamiltonian is (and this is the Bogoliubov de Gennes Hamiltonian with energies relative to $E_\mathbf{k}$):
\begin{equation}
H_\text{BdG}\left(\mathbf{k} \right) = \begin{pmatrix}
\xi_\mathbf{k} &  - \Delta_\mathbf{k}\\
- \Delta^*_\mathbf{k} & - \xi_\mathbf{k}
\end{pmatrix} \label{eq:H-BdG}
\end{equation}
%For this, the commutation relation for $H_\text{BCS}$ and $c^\dagger_{\mathbf{k}, \uparrow}$, $c_{- \mathbf{k}, \downarrow}$ have been used.
This Hamiltonian form eq. (\ref{eq:H-BdG}) has the eigenvalues
\begin{equation}
 \pm E_\mathbf{k} = \pm \sqrt{\xi_\mathbf{k}^2 + |\Delta_\mathbf{k}|^2  }.
\end{equation}
%Include eigenstates?
To finally arrive at the spatially dependent form of eg. (\ref{eq:H-BdG}), the Hamiltonian is Fourier-transformed.
\begin{eqnarray}
H_\text{BdG} \left( \mathbf{r} \right) &:=& \frac{1}{N} \sum_\mathbf{k} e^{i \mathbf{k \cdot r}} H_\text{BdG}\left( \mathbf{k} \right) \\
&=& \begin{pmatrix}
H_0\left( \mathbf{r} \right)  &  - \Delta \left( \mathbf{r} \right) \\
- \Delta^* \left( \mathbf{r} \right)  & - H_0 \left( \mathbf{r} \right) 
\end{pmatrix} \label{eq:H-BdG-r}
\end{eqnarray}
$H_0 \left( \mathbf{r} \right) $ is the free Hamiltonian. Corresponding Schr\"odinger equaions are called BdG-equations:
\begin{eqnarray}
H_\text{BdG} \left( \mathbf{r} \right) \Psi\left( \mathbf{r} \right) &=& E \Psi\left( \mathbf{r} \right)\label{eq:BdG-eq} \\
\Psi\left( \mathbf{r} \right)  &=& \begin{pmatrix}
\Psi_1\left( \mathbf{r} \right) \\ \Psi_2\left( \mathbf{r} \right) 
\end{pmatrix}\label{eq:BdG-spinor}
\end{eqnarray}

\section{Andreev reflection- NS interface}\label{sec:NS}

%Andreev approximation? Smooth variation over k_f? Neglecting second order derivatives? Semiclassical approximation?

Now that the principles of BCS theory have been established, the physical effects at the interface between a superconductor and a normal are to be outlined.

The most important detail when modelling the interface between a superconductor and a normal metal is the superconducting order parameter $\Delta \left( \mathbf{r} \right)$. It is present in the superconducting region and zero in a normal metal. To keep the model as simple as possible, a step-like behaviour is assumed. This means that for an interface placed at $x=0$, the superconducting order parameter becomes a function of $x$ and can be written as
\begin{equation}
\Delta \left( x \right) = \theta \left(x \right)
\end{equation}.
How does this model differ from a quantum mechanical step potential set-up? The formalism is virtually identical, but there is a subtle and important difference in the results. In the normal region, there are electrons, whereas in the superconducting regions, there is a condensate of Cooper pairs. A normal electron can be reflected at the interface as a hole and an additional Cooper pair can be created in the superconducting region.

By solving the Bogoliubov-de-Gennes equation in (\ref{eq:BdG-eq}), this picture becomes clearer.  This equation needs to be solved both for the normal and the superconducting region. When treating this problem quantum-mechanically, energies below and above the gap need to be considered independently. The resulting wave functions have to be continuous at the interface. Depending on the region, the gap parameter in the Hamiltonian in eq. (\ref{eq:H-BdG-r}) is either zero or $\Delta_0$.

\subsection{Solution for the normal region}\label{sec:normal-region}
In the normal region, for $x < 0$, only the free particle Hamiltonian $H_0$ has to be considered. With $\hbar = 1$, it reads
\begin{equation}
H_0 = - \frac{\hbar^2}{2m} \nabla^2 - \mu 
\end{equation}
since no additional (vector) potential is present. The solution is simply a superposition of plane waves with the corresponding wave vectors
\begin{equation}
k_{1/2}^2 = k_F^2 \pm 2mE \label{eq:k-normal}.
\end{equation}

\subsection{Solution in the superconducting region}
 
If the \textbf{energy is above the gap} $\mathbf{ |E| > \Delta_0}$, then the solution is, again, a superposition of plane waves, similar to section \ref{sec:normal-region}. But this time, the gap parameter $\Delta_0$ couples the amplitures of the wave functions.
\begin{equation}
\Psi_2 \left( \mathbf{r} \right)  = \frac{1}{\Delta_0} \left( E + \frac{1}{2m} \nabla^2 + \mu \right) \Psi_1 \left( \mathbf{r}  \right) \label{eq:psi-coupling}
\end{equation}
The corresponding wave vectors are 
\begin{equation}
q_{1/2}^2 = 2m \left( \mu + \sqrt{E^2 - \Delta_0^2} \right)
\end{equation}
\newline
For \textbf{energies below the gap} $\mathbf{ |E| < \Delta_0}$, the a separable ansatz for $\Psi_1$ is chosen as
\begin{equation}
\Psi_1 \left( \mathbf{r} \right)  = e^{i ( \mathbf{k}_{1 \parallel} \cdot \mathbf{r}_{\parallel} ) } \Phi_1\left( x \right),
\end{equation}
where $\mathbf{k}_{1 \parallel} = (k_{1y}, k_{2y})$. This leads to the effective one-dimensional differential equation for $\Phi_1\left( x \right)$.
\begin{equation}
\left( - \frac{k_{1, \parallel}^2}{2m} + \frac{1}{2m} \frac{d^2}{dx^2} + \mu \right)^2 \Phi_1\left( x \right) = (E^2 - \Delta_0^2) \Phi_1\left( x \right) \label{eq:phi-diff}
\end{equation}
To solve the differential equation in (\ref{eq:phi-diff}), an exponential ansatz is chosen:
\begin{equation}
\Phi_1(x) = e^{\kappa x + i q x}.
\end{equation}
The chosen ansatz then yields
\begin{eqnarray}
\kappa^2 &=& k_{1\parallel}^2 + q^2 - 2m\mu \\
- \frac{\kappa^2 q^2}{m} &=& E^2 - \Delta_0^2.
\end{eqnarray}
The explicit solutions of $q$ and $\kappa$ are given by
\begin{eqnarray}
q^2 &=& \frac{1}{2}\left( k_F^2 - k_{1, \parallel}^2 \pm \sqrt{ (k_F^2 - k_{1, \parallel}^2) + 4m(\Delta_0^2 - E^2)} \right) \label{eq:q2}\\
\kappa^2 &=& \frac{1}{2}\left( k_{1, \parallel}^2 - k_F^2 \pm \sqrt{ (k_F^2 - k_{1, \parallel}^2) + 4m(\Delta_0^2 - E^2)} \right)\label{eq:kappa2}.
\end{eqnarray}
These solutions have to be checked for physical plausibility. Since a real value for $q$ has been assumed, only the solution with positive sign in the square root of eq. (\ref{eq:q2}) is possible. Considering eq. (\ref{eq:kappa2}), only the negative solution is relevant, because the positive solution would grow exponentially for $x \rightarrow \infty$.
\begin{eqnarray}
q &=& \pm q_1 := \pm \frac{1}{\sqrt{2}}\sqrt{\left( k_F^2 - k_{1, \parallel}^2 \pm \sqrt{ (k_F^2 - k_{1, \parallel}^2) + 4m(\Delta_0^2 - E^2)} \right)} \label{eq:q} \\
\kappa &=& - \kappa_1 := \pm \frac{1}{\sqrt{2}}\sqrt{\left( k_{1, \parallel}^2 - k_F^2 \pm \sqrt{ (k_F^2 - k_{1, \parallel}^2) + 4m(\Delta_0^2 - E^2)} \right)}\label{eq:kappa}
\end{eqnarray}
Up to this point, we have only considered the differential equation for $\Phi_1\left( x \right)$. As a result, an expression for the wave function $\Psi_1 \left( \mathbf{r} \right)$ has been found. The components of $\Psi \left( \mathbf{r} \right)$, $\Psi_1 \left( \mathbf{r} \right)$ and $\Psi_2 \left( \mathbf{r} \right)$, are coupled by eq. (\ref{eq:psi-coupling}). Solving the differential equation for $\Phi_2 \left( x \right)$ will therefore lead to the same results. In other words, since $\mathbf{k_{1, \parallel}} = \mathbf{k_{2, \parallel}}$, it follows directly that $\kappa_1 = \kappa_2$ and $q_1 = q_2$. 
Combining the results, we find 
\begin{equation}
\Psi_2 \left( x \right) = \frac{E \mp i \sqrt{\Delta_0^2 - E^2}}{\Delta_0} \Psi_1 \left( x \right).
\end{equation}
The next step in solving the problem for the NS-interface is matching the solutions at both sides of the interface. The ansatz for the wave function in the normal region reads
\begin{equation}
\Psi \left( r \right) = \begin{pmatrix}
e^{i \mathbf{k}_1 \mathbf{r}} + r \cdot e^{i \tilde{\mathbf{k}}_1 \mathbf{r}} \\
a \cdot e^{i \mathbf{k}_2 \mathbf{r}}
\end{pmatrix} \quad \text{for ~} x \leq 0,
\end{equation}
\begin{eqnarray}
\mathbf{k}_1  = \begin{pmatrix} ~k_{1, x} \\ ~k_{1, y}\\ ~k_{1, z} \end{pmatrix}, \quad \tilde{\mathbf{k}}_1  = \begin{pmatrix} -k_{1, x} \\ ~k_{1, y}\\ ~k_{1, z} \end{pmatrix}, \quad
\mathbf{k}_2  = \begin{pmatrix} ~k_{2, x} \\ ~k_{1, y}\\ ~k_{1, z} \end{pmatrix}, \quad k_{2, x} > 0,
\end{eqnarray}
where $\mathbf{k}_1$ is the wave vector of the incoming electron and $\tilde{\mathbf{k}}_1$ corresponds to a reflected electron with amplitude $r$. The wave vector $\mathbf{k}_2$ however, corresponds to the second spinor component which was defined in eq. (\ref{eq:spinor}) as $\ket{Psi_{\mathbf{k}2}} =  c_{- \mathbf{k}, \downarrow} \ket{\psi_\text{BCS}}$. This means that it corresponds to a hole in the normal region. \\
Rewriting eq. (\ref{eq:k-normal}) makes clear that this hole travels almost exactly into the opposite direction as the electron.
\begin{eqnarray}
k_{1, x}^2 + k_{1, \parallel}^2 &=& k_F^2 + 2mE \\
k_{2, x}^2 + k_{1, \parallel}^2 &=& k_F^2 - 2mE.
\end{eqnarray}
Since the energy is below the superconducting gap, it holds that $E < \Delta_0 \ll \mu = \frac{k_F^2}{2m}$. From equation (\ref{eq:k-normal}), it can be seen that $|k_{2x} - k_{2x}|$ is small and $\mathbf{k}_{1\parallel} = \mathbf{k}_{2\parallel}$: The incoming electron can be reflected as a hole travelling into the opposite direction. Still, it has to be shown that this interpretation is valid and therefore the wave functions have to be matched at the interface.
\newline
An ansatz for the wave function in the superconducting region uses the results of eq. (\ref{eq:phi-diff}). The wave function is proportional to $\Phi \left( x \right) = e^{-\kappa_1 x } \left( e^{+i q_1 x} + e^{- i q_1 x} \right)$ (see eq. (\ref{eq:q}) and (\ref{eq:kappa})).
\begin{equation}
\Psi \left( \mathbf{r} \right) = e^{- \kappa_1 x }  \begin{pmatrix} \alpha_+ e^{i ( q_1x + \mathbf{k}_\parallel \cdot \mathbf{r}_\parallel)} + \alpha_- e^{i ( - q_1x + \mathbf{k}_\parallel \cdot \mathbf{r}_\parallel)}\\ \beta_+ e^{i ( q_1x + \mathbf{k}_\parallel \cdot \mathbf{r}_\parallel)} + \beta_-e^{i ( - q_1x + \mathbf{k}_\parallel \cdot \mathbf{r}_\parallel)} \end{pmatrix}
\end{equation}
Requiring continuity of the wave function  $\Psi \left( \mathbf{r} \right)$ and its derivative gives 
\begin{eqnarray}
1 + r &=& \alpha_+ + \alpha_- \\
a &=& \beta_+ + \beta_- \\
i k_{1x}(1-r) &=& \alpha_+ (-\kappa_1 + i q_1) + \alpha_- ( -\kappa_1 - i q_1) \\
i k_{2x} a &=& \beta_+(-\kappa_1 + i q_1) + \beta_- ( -\kappa_1 - i q_1). \\
\end{eqnarray}
The coupling of the amplitudes of $\Psi_1 \left( \mathbf{r} \right)$ and $\Psi_2 \left( \mathbf{r} \right)$ by eq. (\ref{eq:psi-coupling}) eventually yields
\begin{equation}
\beta_\pm = \frac{E \mp i \sqrt{\Delta_0^2 - E^2}}{\Delta_0}\alpha_\pm.
\end{equation}
These equations above build a linear system, which can be solved. But the most important insight comes from the reflection coefficient $r$. When evaluating the expression, one can see that $|r|^2 < 1$, which means that not all electrons are reflected. This is the important difference to a step-like potential.

\section{Theory of SNS junction}
%Again, start with a picture, explain what happens in the picutre --> Andreev Bound states
So far, only NS interfaces have been considered. The same procedure can be applied to superconductor - normal metal - superconductor (SNS) junctions: A sandwich structure of a superconductor on the left side, a normal region in the middle and a superconductor on the right side. At both interfaces, a particle can be Andreev-reflected.
Each time an electron is Andreev reflected at the right side and a hole travels back, a cooper pair is induced into the right superconductor. In the same manner, a cooper pair is stolen from the left superconductor when the hole is Andreev reflected as an electron. As an overall consequence, a supercurrent through the SNS junction can be observed. This process leads to bound states, the so called Andreev bound states. \\
%Write down the ansatz and explain, that it is similar to above, maybe qualitative results?

The wave functions for this set-up are similar to those in section (\ref{sec:NS}). In the normal region, not only an incoming electron from the left, but also a hole travelling to the right side of the superconductor have to be considered:
\begin{equation}
\Psi_N\left(x\right) = \begin{pmatrix}
A_{e\rightarrow} e^{i k_1 x} + A_{e\leftarrow} e^{- i k_1 x} \\
A_{h\leftarrow}e^{i k_2 x}  + A_{h\rightarrow} e^{-i k_2 x} 
\end{pmatrix}.
\end{equation}
The situation for the superconducting region is the same as in section (\ref{sec:NS}) with now two superconducting regions and therefore two superconducting wave functions $\Psi^S_L \left( x \right)$ and $\Psi^S_R \left( x \right)$. They differ in their amplitudes, and each side has a different superconducting phase $\chi_R$ and $\chi_L$.
\begin{eqnarray}
\Psi^S_R \left( x \right) &=& e^{i \frac{\chi_R}{2}} \left(  e^{i k^+ x } \begin{pmatrix} C_{e\rightarrow} \\ D_{e\rightarrow} \end{pmatrix} + e^{- i k^- x } \begin{pmatrix} C_{h\rightarrow} \\ D_{h\rightarrow} \end{pmatrix} \right) \\
\Psi^S_L \left( x \right) &=& e^{i \frac{\chi_L}{2}} \left(  e^{i k^- x } \begin{pmatrix} A_{h\leftarrow} \\ B_{h\leftarrow}\end{pmatrix} + e^{- i k^+ x } \begin{pmatrix} A_{e \leftarrow}\\ B_{e \leftarrow}\end{pmatrix} \right) 
\end{eqnarray}
%TODO explain k+, k-, k1, k2...

Since the calculations for the SNS junctions follow the same scheme as for NS interfaces in section 
%TODO
, intermediate steps are omitted and the solution presents itself as follows:
%TODO

%Bohr sommerfeld quantization rule?


%How is this analogoous to Josephson Junction
%Maybe merge with foundations chapter?



\section{Specular Andreev reflection, graphene specifics}
An unusual form of Andreev reflection has been found on graphene - superconductor interfaces %TODO cite.
In the previously discussed Andreev retro reflection, the electron and the reflected hole both lie in the conductance band. In more exotic materials, like  single layer graphene (SLG), the reflected hole undergoes a transition to the valence band. In this case, the reflection angle is a function of $E_F$, and for $E_F = 0$, the reflection is specular. 

%In Graphene --> Dirac cone, linear energy dispersion --> constant velocity of electrons. Interesting phenomenon occurs: specular Andreev relfection. Close to the Dirac point (where conduction and valence band touch), an electron from conduction band is converted into a hole from valence band by the superconductor.
%Explain process, pictures, can be seen in differential conductance
%(NOrmal Andreev retro reflection: electron and hole both lie in conductance band, when reflected)
%TODO plot of the differential conductance
%TODO Zusammenhang mit E_F ob Specular AR oder Retro AR
%Bilayer graphene: crossover from retro to specular AR in bilayer graphene
%"We find a characteristic signature of the crossover from intraband retro (high E F ) to interband specular (low E F ) ARs that manifests itself in a strongly suppressed interfacial conductance when the excitation energy |ε| = |E F | < (the SC gap). The sharpness of these conductance dips is strongly dependent on the size of the potential step at
%the BLG/SC interface U 0 ." \cite{Efetov2016}
 