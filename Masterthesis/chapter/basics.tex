\chapter{Things you have to know before you read my thesis...}
\label{ch:basics}

The interface between a superconductor and a normal metal can be described with an effective 1D Bogoliubov-de-Gennes Hamiltonian, describing the superconducting order parameter with a step-like function as order parameter
\begin{eqnarray}
\begin{pmatrix}
-\frac{\hbar^2}{2m} \nabla^2 - \mu & \Delta(x) \\
\Delta^*(x) & \frac{\hbar^2}{2m} \nabla^2 + \mu
\end{pmatrix}
\begin{pmatrix}
u(x) \\
v(x)
\end{pmatrix} = E 
\begin{pmatrix}
u(x)\\
v(x)
\end{pmatrix}, \quad \Delta(x) = \Delta_0 e^{i \phi} \theta(x)
\end{eqnarray}
This is the so called Blonder-Tinkham-Klapwijk model and can be solved seperately for each of side of the NS interface. For the normal side, the superconducting order parameter vanishes and the equation leads to two solutions
\begin{eqnarray}
\psi_e^{\pm}(x) &=& \begin{pmatrix} 1 \\ 0 \end{pmatrix} e^{\pm i k_e x}, \quad k_e = k_F \sqrt{1 + \frac{E}{e_F}}\\ 
\psi_h^{\pm}(x) &=& \begin{pmatrix} 0 \\ 1 \end{pmatrix} e^{\pm i k_h x}, \quad k_h =  k_F \sqrt{1 - \frac{E}{e_F}}
\end{eqnarray}
\textbf{TODO:} electron solution because of term $e_f + E$, hole solution because of $e_f - E$ ? \\
To find the wavefunctions for the superconducting side, we have to distinguish between propagating waves ($E > \Delta_0$) and 