\chapter{Introduction}
\label{ch:introduction}

% Zun\"achst: transport in graphene, warum cool
First successfull transport experiments in graphene \cite{Zhang2005}, \cite{Novoselov2005}

First thing that is weird about (gapless) graphene: it is described by a relativistic, massless Dirac electrons. Experiments \cite{Novoselov2005} show that electronic transport is goverened by Dirac equation, charge carriers have an effective "speed of light". This paper shows unusual effects that are characteristic for two-dimensional Dirac fermions: 1) Conductivity in graphene never falls below a minimum value (even when charge carrier density goes to zero) 2) anomalous integer quantum hall effect (occurs at half integer filling factors)
Transport measurements in graphene: integer quantum hall effect, weak localization, 

\cite{Zhu2017}: Spectra of MLG and BLG graphene are gapless and protected by symmetry of crystal lattice. When symmetry is broken by interaction with substrate or applying electric field: energy gap opens. In BLG: energy gap can be controlled by displacement field $\mathbf{D}$. Controlled induction of insulating state in graphene \cite{Oostinga2008}

\cite{Zhu2017}: Manchester group, proximity induces superconductivity: high gap energies (big gap), but still there is low resistivity (not the expected insulating state). Conductive channels were found an explained for both MLG and BLG at the Charge neutrality point and were explained by valley currents, zero energy edge states. 

supercurrent at charge neutrality point porpagates along edge channels, shunts the insulating bulk in graphene. Question adressed: in graphene, large gaps can be opened, but they rarely lead to an highly insulating state which would be expected at low temperatures. Idea: edge currents are responsible for transport, edge conductance due to nontrivial topology of gapped Dirac spectrum: 1) electronic states due to zigzag segments. 2) valley Hall effect ?
%Übergang zu Supraleitung und Fraunhofer paatern

Why study Fraunhofer patterns? By measuring critical current as a function of perpendicular B, the local current density in x direction perpendicular to current flow can be deduced \cite{Dynes1971}. By variying top and bottom voltage it is possible to keep BLG graphene charge neutral while doping the two graphene layers with the opposite sign. This results into the displacement field which translates directly into the gap.At high doping ($E_F > E_\text{gap}$, the critical current depends weakly on the displacement field.

\cite{Heersche2007}: Josephson effect in mesoscopic junctions: charge density in the graphene layer can be controlled by a gate electrode. Observation of of a supercurrent that (depending on gate voltage) is carried either by electrons in the conduction band or by holes in the valence band. finding of a finite normal state conductance and finite supercurrent at zero charge density

%Zusammenfassung: supercurrent ist geil und kann genutzt werden, um den Stromfluss zu untersuchen. Aber die Frage  ist doch eigentlich: kann man die Amplitude und das confinement gleichzeitig ver\"andern? Siehe dazu das QPC gate!
