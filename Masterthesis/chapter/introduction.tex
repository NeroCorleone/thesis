\chapter{Introduction}
\label{ch:introduction}

\section*{Graphene, Superconductivity, And Quantum Point Contacts}
In the year 1947, when graphene was exclusively theoretically investigated and its transport properties predicted by Phil Wallace \cite{Wallace1947}, the possibility of its existence was still very controversial. The Mermin-Wagner theorem prohibited the existence of long-range order in low-dimensional systems \cite{Mermin1966}.

It was therefore considered a scientific breakthrough when today's Nobel Prize winners Andre Geim and Konstantin Novoselov were able to successfully produce graphene in Manchester in 2004 \cite{Novoselov2004}. With the first successful transport measurements, \cite{Zhang2005} and \cite{Novoselov2005}, the previously predicted relativistic nature of charge carriers \cite{Semenoff1984} could be confirmed. The measurement of the quantum Hall effect in graphene revealed a half-integer re-occurance of the Hall plateaus. This result confirmed that the charge carriers in graphene behave like Dirac electrons.

The relativistic property of the Dirac electrons is shown in the spectrum: Near the K-points, the corner points of the first Brillouin zone, graphene has a linear dispersion relation. This linear behaviour leads to a very good charge carrier mobility. The graphene spectrum is still noticeable because valence and conduction bands touch each other at the K points -- graphene thus forms a semiconductor with a disappearing band gap. 

The extraordinary electronic conduction properties, the impressive tensile strength combined with a very low substance make graphene an interesting candidate for application in the computer industry \cite{Jurewicz2014} or, for example, in battery technology \cite{Son2017}.

Its properties as a semiconductor are of particular importance in industrial applications, but also in fundamental research. In order to enable application in complex electronic circuits, it is necessary to be able to influence the conductivity properties locally. This is difficult to do in single layer graphene. The conductivity never drops below a certain value of $e^2/h$, and it is not possible to confine the charge carriers in graphene \cite{Katsnelson2006}. Double-layer graphene (BLG) promises to be a solution to this problem. When BLG is exposed to an electrostatic field, the two layers each have a slightly different potential. This means that a band gap can be opened in the BLG spectrum, depending on the strength of the electric field. This suggests that an isolating state could be achieved during transport measurements with BLG.

In a superconductor-BLG-superconductor junction -- a junction consisting of a BLG surface contacted at the sides with two superconductors -- the expected insulating state could not be found in transport measurements \cite{Zhu2017}. By measuring the critical current $I_c$ as a function of the applied magnetic field $B$, the distribution of the current density within the sample can be found. Based on the results of the current density distribution, it was assumed that current transport through channels at the edges of the sample is responsible for the finite conductivity of the sample. 

At the Institute of Nanotechnology in the research group of Ralph Kruppke, superconductor-BLG-superconductor junctions are investigated with a so-called \emph{weak link}, i. e. a constriction in the sample through which the current must pass. This weak link has the form of a quantum-point-contact (QPC) and its experimental investigation is supported by an analytical study, which is presented in this thesis. The present work deals with the question of how the experimental data can be explained in the context of a quasi-classical transport theory. To this end, existing approaches have been reviewed and an adjusted approach has been proposed for the QPC set-up. The results of the quasi-classical analysis of the QPC, along with the experimental data and numerical simulations, have been published in \cite{Kraft2017}.

\section*{Structure Of The Thesis}

This thesis deals with the physics of such superconductor-normal metal-superconductor junctions (SNS junctions). The supercurrent for the QPC structure is explored within the framework of this quasi-classical transport theory. The results of the investigation are supplemented by the calculation of the supercurrent using the tight-binding method.

In chapters \ref{ch:basics} and \ref{ch:basics-numerical} the basics necessary for understanding this work are presented. In chapter \ref{ch:basics} processes that are relevant at the interface of superconductors and normal conductors are examined in more detail. The Andreev reflection should be mentioned in particular. Based on this, it is explained how current flow can occur in an SNS junction. Chapter \ref{ch:basics-numerical} discusses details of the tight binding method and explains the scattering matrix approach. First, the tight-binding Hamiltonian for graphene and BLG is derived and then the model Hamiltonian for the QPC system is explained. 

The experimental questions relevant to this work are introduced in chapter \ref{ch:experiment}. The experimental findings of the SNS junction with a QPC structure for both the normal state and the superconducting state are presented and discussed.

The results for the supercurrent obtained within the framework of quasi-classical transport theory are presented in chapter \ref{ch:analyticalmodel}. A key result is that the transition from an oscillating to a Gaussian bell-shaped pattern is found in accordance with the experimental findings. Furthermore, the current transport along the edges of the sample is investigated within the framework of this transport theory: The calculations show how the critical current is influenced by boundary channels, and also that transport along these channels is unlikely in this particular configuration.

In chapter \ref{ch:numerical-results} a wave-guide-shaped set-up with tight-binding models is investigated in addition to the QPC set-up. The calculated results for supercurrent and conductivity are presented, and the influence of disorder in the sample and defects at the edges are discussed. The calculated results for the QPC set-up are well in line with the experimental findings.
