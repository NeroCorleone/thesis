\chapter{Conclusion And Outlook}
\label{ch:conclusion}

\subsection*{Results}
In this thesis the supercurrent in the QPC set-up has been studied within the quasi-classical framework and numerically using \texttt{kwant}. Apart from that, two more set-ups have been modelled numerically and their transport properties have been evaluated.

The central result of this work is that
\begin{equation}
J(\tilde{\chi}(y_1, y_2), \phi) \propto \int \int_{-W/2}^{~W/2} dy_1 dy_2 \frac{ \mathcal{J}(\tilde{\chi}(y_1, y_2)) }{ \left( 1 - \left(\frac{y_1 - y_2}{L}\right)^2 \right)^2 }
\end{equation}
The denominator for a current through SNS junction goes with a power of 2. This contradicts the result of \cite{Barzykin1999}. 
The current density for the QPC has been established, it reads
\begin{equation}
J(\tilde{\chi}(y_1, y_2), \phi) \propto \int \int_{-W/2}^{W/2} d y_1 d y_2 \frac{\cos \left( \frac{\pi \phi}{W}(y_1 + y_2) \right)}{\left[ 1 + \left(\frac{y_1 - y_2}{L}\right)^2\right]^2} \label{eq:josephson_current},
\end{equation}
The critical current was evaluated for the $\phi \rightarrow 0$ boundary case where a parabolic function dependent on W/L was found for the QPC. In the other boundary case, $\phi \rightarrow \infty$, the exponential decline for large $\phi$ is correctly reproduced. The behaviour of the boundary currents was also investigated within the framework of the quasi-classical approximation. It is shown that for different coefficients of transmittance $\mathcal{T}_e$ of the edge and $\mathcal{T}_q$ of the QPC the critical current is modulated by a factor 
\begin{equation}
\mathcal{C} =  \frac{| \mathcal{T}_q / \mathcal{T}_e + \cos \left( \pi \phi \right)/2 |}{\mathcal{T}_q / \mathcal{T}_e + 1/2}.
\end{equation}
This modulation is already visible with values of around $\mathcal{T}_e/ \mathcal{T}_q = 1 / 100$. This is a strong indication that there are no edge streams in the observed data. 

The calculations are supported by numerical simulation of the set-up with \texttt{kwant}. In the simulations, the boundary case $\phi \rightarrow \infty$ also shows the exponential decay for the regime in which current transport is limited to a few channels. In addition to the QPC set-up, other set-ups such as the wave-guide system and the half-barrier system reflect the experimentally measured behaviour well. The experimental data has not yet been published, however.

 
\subsection*{Outlook}

The transport theory presented in the scope of this thesis can be expanded in various aspects. It was demonstrated that this theory can model a \emph{weak link} in the form of a QPC fairly well. It is therefore conceivable to model other, similar forms of weak links.

In this thesis, a short SNS-set-up was considered. As a consequence, certain approximations that simplify the calculations have been made.  For the other boundary case of a long SNS junction, scattering at the edges is an important component for current transport and must therefore be taken into account. Secondly, higher-order terms must be considered when calculating the current density. Thus, the simplest form of the Josephson equation is no longer sufficient in this case. 

So far, only Andreev retro-reflexion has been considered, where the hole is reflected on approximately the same trajectory as the electron, only with inverted impulse. In the case of graphene, however, the process is dominated by the specular Andreev reflection, in which the reflected hole does not return on the same trajectory. The quasi-classical theory can also be extended to this case and thus consider graphene-specific properties.

The numerical simulations with \texttt{kwant} can be extended. This thesis work implements the BLG system as a de facto two-dimensional system. However, realistic experimental set-ups are more complex: On the one hand, they are three-dimensional, and on the other hand, the assumptions of the spatial distribution of electrostatic fields so far are too simplistic. A combination of calculating these fields with numerical methods and simulations with \texttt{kwant} will allow for a more complete modelling of the experiments. 







