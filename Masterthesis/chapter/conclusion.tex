\chapter{Conclusion And Outlook}
\label{ch:conclusion}
%%%%%%%%%%%%%%%%%%%%%%%%%%%%
%Zusammenfassung der Arbeit%
%%%%%%%%%%%%%%%%%%%%%%%%%%%%

%Meine Arbeit hat sich mit SNS junctions und der quasiklassischen Beschreibung beschäftigt. Ein weiterer Punkt ist die numerische Simulation von konkreten Experimenten. Ich konnte zeigen, dass die Ergebnisse von meiner


%%%%%%%%%%%%%%%%%%%%%%%%%%%%
%		Ergebnisse?		   %
%%%%%%%%%%%%%%%%%%%%%%%%%%%%
%Es wurden Ansätze für den Transport überprüft und auf den konkreten Fall des QPC übertragen (neu entwickelt, Formel). Dieser Ansatz ist anders als in Glazman, Zagoskin etc.
%Die quasiklassische Transporttheorie wurde erfolgreich auf das QPC Setup angewendet. Es konnte gezeigt werden, dass sowohl das Limit phi -> unendlich als auch phi-> 0 korrekt wiedergegeben werden. Weiterhin konnte gezeigt werden, dass das Verhalten der Kantenströme gut beschrieben werden kann.

%Auf das Experiment eingehen, was wird gut beschrieben, was nicht? Was steht 

%Eine Berechung der kritischen Stroms im Rahmen der Streumatrix-Theorie wurde implementiert. Die Tendenz ist, dass die QPC und HB Experimente richtig wiedergegeben werden. Das Waveguide Setup konnte simuliert werden, daraus konnte, weil es der tolle Fall mit translationsinvariantem System ist, die Stromdichte mit der Dyson Fuller Methode berechnet werden. 

%%%%%%%%%%%%%%%%%%%%%%%%%%%%
%		 Ausblick		   %
%%%%%%%%%%%%%%%%%%%%%%%%%%%%

%Analytische Rechnungen können für weitere Setups berechnet werden, für kompliziertere Setups. Dabei kann beispielsweise Mehrfachstreuung und Anisotrope Streeuung hinzugezogen werden. 

%Im Kontext auf die Experimente sind die Möglichkeiten beinahe unbeschränkt. Es gibt Ansätze, für Spin Orbit Coupling in BLG, die nutzen und afwändige experimentelle Setups beschreiben.



