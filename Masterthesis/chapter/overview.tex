%Modify overview chapter
\makeatletter% siehe De-TeX-FAQ
\renewcommand*{\chapterformat}{%
\begingroup% damit \unitlength-Änderung lokal bleibt
\setlength{\unitlength}{1mm}%
\begin{picture}(0,40)(0,5)%
\setlength{\fboxsep}{0pt}%
%\put(0,0){\framebox(20,30){}}%
%\put(0,20){\makebox(20,20){\rule{20\unitlength}{20\unitlength}}}%
\put(0,15){\color{black}\linethickness{0.4pt}\line(1,0){\dimexpr\textwidth-0\unitlength\relax\@gobble}}% line
%\put(0,0){\makebox(18,20)[r]{\fontsize{20\unitlength}{1}\color{white}\selectfont\thechapter %\kern-.04em% Ziffer in der Zeichenzelle nach rechts schieben
%}}%
\put(20,15){\makebox(\dimexpr \textwidth-20\unitlength\relax\@gobble,\ht\strutbox\@gobble)[l]{%\ %\normalsize\color{black}\chapapp~\thechapter\autodot 
}}%
\end{picture} % <- Leerzeichen ist hier beabsichtigt!
\endgroup
}%
\makeatother

\newcommand{\nocontentsline}[3]{}
\newcommand{\tocless}[2]{\bgroup\let\addcontentsline=\nocontentsline#1{#2}\egroup}

\setcounter{chapter}{-1}
\tocless\chapter{Overview}
\addcontentsline{toc}{chapter}{Overview}
