%package file 
\usepackage[english]{babel}	%mehrsprachiges Dokument
\usepackage[numbers,square,comma,sort&compress]{natbib}
\bibliographystyle{unsrtnat}

\usepackage{float}			%erm�glicht flexiblere Bild- und Tabellenpositionierung
\usepackage{graphicx}
\usepackage[absolute,overlay]{textpos} %textblock env for text positioning
\usepackage{wrapfig}
\usepackage{tabularx}
\usepackage{multirow}
%Feynman Diagrams 
%\usepackage{feynmp}

\usepackage{amsmath} %Zus�tzliches Mathepaket, dass die Align-Umgebung beinhaltet
\usepackage{physics}

\usepackage{color}

\usepackage{caption}
\usepackage{subcaption}

\usepackage{units}

\usepackage{rotating}

\usepackage{hyperref}

%%%%%%%%%%%%%%%%%%%%%%%%%%%%%
%Globale Layouteinstellungen%
%%%%%%%%%%%%%%%%%%%%%%%%%%%%%
\setlength{\headheight}{14.5pt}              % H�he der Kopfzeile
\setlength{\textheight}{21.0cm}              % Texth�he
\setlength{\textwidth}{16.0cm}               % Textbreite
\setlength{\topmargin}{1.0cm}                % oberen Seitenrand festlegen, bzw. verschieben
\setlength{\oddsidemargin}{0.2cm}            % Seitenrand anpassen -> ungerade Seiten !
\setlength{\evensidemargin}{-0.2cm}          % Seitenrand anpassen -> gerade Seiten !
\setcounter{secnumdepth}{5}                  % Schachtelungstiefe Ueberschriften
\setcounter{tocdepth}{5}                     % Schachtelungstiefe Eintraege im Inhaltsverzeichnis
\setlength{\parindent}{0pt}                  % Einzug bei neuen Abs�tzen
\setlength{\parskip}{5pt plus 2pt minus 1pt} % Abstand zwischen zwei Abs�tzen
\renewcommand{\baselinestretch}{1.1}         % Ver�nderung des Zeilenabstandes {Faktor}