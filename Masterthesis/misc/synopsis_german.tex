\section*{Deutsche Zusammenfassung}
\subsection*{Motivation}

Als Graphen theoretisch untersucht und seine Transporteigenschaften vorhergesagt wurden, war die M\"oglichkeit seiner Existenz noch sehr umstritten. Das Mermin-Wagner-Theorem verbot die Existenz von langreichweitiger Ordnung in niedrigdimensionalen Systemen \cite{Mermin1966}.

Es war daher ein wissenschaftlicher Durchbruch, als 2004 die heutigen Nobelpreistr\"ager Andre Geim und Konstantin Novoselov in Manchester Graphen erfolgreich Herstellen konnten \cite{Novoselov2004}. Mit den ersten erfolreichen Transportmessungen, \cite{Zhang2005} und \cite{Novoselov2005}, konnte die zuvor vorhergesagte relativistische Natur der Ladungstr\"ager \cite{Semenoff1984} best\"atigt werden. Die Messung des Quanten-Hall-Effekts in Graphen ergab eine halbzahlige Wiederkehr der Hall Plateaus. Dieses Ergebnis best\"atigte, dass sich die Ladungstr\"ager in Graphen wie Dirac-Elektronen verhalten.
Die relativistische Eigenschaft der Dirac-Elektronen zeigt sich im Spektrum. In der N\"ahe der K-Punkte, den Eckpunkten der ersten Brillouin-Zone, weist Graphen eine lineare Dispersionsrelation auf. Dieses lineare Verhalten f\"uhrt zu sehr guten Ladungstr\"agermobilit\"at. Das Graphen-Spektrum ist weiterhin auff\"allig, weil sich Valenz- und Leitungsband an den besagten K-Punkten ber\"uhren -- Graphen bildet somit einen Halbleiter mit verschwindender Bandl\"ucke. 

Die au{\ss}ergew\"ohnlichen elektronischen Leitungseigenschaften, die beeindruckenden Zugfestigkeit kombiniert mit einer sehr geringen Fl\"achenmasse macht Graphen zu einem interessaten Kandidaten f\"ur die Anwendung in der Computerindustrie \cite{Jurewicz2014} oder beispielsweise in er Batterietechnik \cite{Son2017}.

Seine Eigenschaften als Halbleiter sind von besonderer Bedeutung in industriellen Anwendungen, aber auch in der Grundlagenforschung. Um die Anwendung in komplexen elektronischen Schaltungen zu erm\"oglichen muss es m\"oglich sein, die Leitungseigenschaften lokal zu kontrollieren. In einer einlagigen Schicht von Graphen gestaltet sich dies schwierig. Die Leitf\"ahigkeit sinkt niemals unter einen bestimmten Wert von $e^2/h$ und es ist nicht m\"oglich, die Ladungstr\"ager in Graphen einzuschr\"anken \cite{Katsnelson2006}. F\"ur dieses Problem verspricht doppellagiges Graphen (BLG) Abbhilfe. Wenn BLG einem elektrostatischen Feld ausgesetzt wird, haben die beiden Schichten jeweils ein leicht unterschiedlisches Potential. Das f\"uhrt dazu, dass sich im BLG-Spektrum eine Bandl\"ucke \"offnen l\"asst, abhh\"angig von der St\"arke des elektrischen Feldes. Das legt die Vermutung nahe, dass bei Transportmessungen mit BLG ein isolierender Zustand erreicht werden k\"onnte.

Transportmessung an Supraleiter-BLG-Supraleiter-Schaltungen -- eine Schaltung bestehend einer BLG-Fl\"ache, die an den Seiten mit zwei Supraleitern kontaktiert ist -- finden nicht den erwarteten isolierenden Zustand \cite{Zhu2017}. Durch Messung des kritischen Stroms $I_c$ in Abh\"angigkeit des angelegten Magnetfeldes $B$ konnte R\"uckschluss auf die Verteilung der Stromdichte innerhalb der Probe getroffen werden. Anhand der Ergebnisse der Stromdichteverteilung wurde die Vermutung aufgestellt, dass Stromtransport durch Kan\"ale an den Kanten der Probe f\"ur die endliche Leitf\"ahigkeit der Probe verantwortlich seien. 

Am Institut f\"ur Nanotechnologie in der Arbeitsgruppe von Ralph Kruppke werden Supraleiter-BLG-Supraleiter-Schaltungen mit einem sogenannten \emph{weak link}, also einer Engstelle in der Probe, durch die der Strom passieren muss, untersucht. Die Experimentelle Untersuchung dieses \emph{weak links}, der die Form eines Quanten-Punkt-Kontaktes hat, wird von einer analytischen Untersuchung des Kontakts im Rahmen einer quasiklassischen Transporttheorie und einer numerischen Untersuchung mittels Tight-binding Methode gest\"utzt. Die Ergebnisse dieser Untersuchung sind in \cite{Kraft2017} ver\"offentlicht. 

Die vorliegende Arbeit besch\"aftigt sich mit der Physik solcher Supraleiter-Normalleiter-Supraleiter-Schaltungen. Der Suprastrom in den Schaltungen wird im Kontext dieser quasiklassischen Transporttheorie untersucht. Ein weiterer Teil der Arbeit ist die Untersuchung unter Zuhilfenahme numerischer Modelle




\subsection*{Gliederung der Arbeit} 
Kapitel \ref{ch:experiment} stellt das Experiment vor, das untersucht wird. Die wichtigsten Ergebnisse werden vorgestellt (normal state conductance, magneto inteferometry...)

In Kaptiel \ref{ch:basics} werden die Grundlagen vorgestellt sind, die notwendig  sind um das quasiklassische Modell zu verstehen.

Kapitel \ref{ch:analyticalmodel} wird die quasiklassische Theorie auf SNS junctions angewendet und im Detail das QPC setup, das auch im Experiment untersucht wird, berechnet. Es wird auf die Bedeutung von Edge strom eingegangen.

Das ganze wird dann auch noch mal numerisch beleuchtet. Die Grundlagen, die zum Verst\"andnis des numerischen Modells be\"otigt werden, sind in Kapitel \ref{ch:basics-numerical} vorgestellt.

In Kapitel \ref{ch:numerical-results} werden dann die Ergbenisse vorgestellt, die aus numerischen Modellen erhalten werden. 

%Schlie{\ss}lich gibt Kapitel \ref{ch:conclusion} einen Uberblick und Ausblick auf die Ergebnisse.

\paragraph{Ergebnisse}
Das sind meine Ergebnisse
\paragraph{Ausblick}
Das sehe ich, wenn ich nach vorne schaue.

