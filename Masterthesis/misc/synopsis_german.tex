\section*{Deutsche Zusammenfassung}
\subsection*{Motivation}

Als Graphen theoretisch untersucht und seine Transporteigenschaften vorhergesagt wurden, war die M\"oglichkeit seiner Existenz noch sehr umstritten. Das Mermin-Wagner-Theorem verbot die Existenz von langreichweitiger Ordnung in niedrigdimensionalen Systemen \cite{Mermin1966}.

Es war daher ein wissenschaftlicher Durchbruch, als 2004 die heutigen Nobelpreistr\"ager Andre Geim und Konstantin Novoselov in Manchester Graphen erfolgreich Herstellen konnten \cite{Novoselov2004}. Mit den ersten erfolreichen Transportmessungen, \cite{Zhang2005} und \cite{Novoselov2005}, konnte die zuvor vorhergesagte relativistische Natur der Ladungstr\"ager \cite{Semenoff1984} best\"atigt werden. Die Messung des Quanten-Hall-Effekts in Graphen ergab eine halbzahlige Wiederkehr der Hall Plateaus. Dieses Ergebnis best\"atigte, dass sich die Ladungstr\"ager in Graphen wie Dirac-Elektronen verhalten.
Die relativistische Eigenschaft der Dirac-Elektronen zeigt sich im Spektrum. In der N\"ahe der K-Punkte, den Eckpunkten der ersten Brillouin-Zone, weist Graphen eine lineare Dispersionsrelation auf. Dieses lineare Verhalten f\"uhrt zu sehr guten Ladungstr\"agermobilit\"at. Das Graphen-Spektrum ist weiterhin auff\"allig, weil sich Valenz- und Leitungsband an den besagten K-Punkten ber\"uhren -- Graphen bildet somit einen Halbleiter mit verschwindender Bandl\"ucke. 

Die au{\ss}ergew\"ohnlichen elektronischen Leitungseigenschaften, die beeindruckenden Zugfestigkeit kombiniert mit einer sehr geringen Fl\"achenmasse macht Graphen zu einem interessaten Kandidaten f\"ur die Anwendung in der Computerindustrie \cite{Jurewicz2014} oder beispielsweise in der Batterietechnik \cite{Son2017}.

Seine Eigenschaften als Halbleiter sind von besonderer Bedeutung in industriellen Anwendungen, aber auch in der Grundlagenforschung. Um die Anwendung in komplexen elektronischen Schaltungen zu erm\"oglichen muss es m\"oglich sein, die Leitungseigenschaften lokal zu kontrollieren. In einer einlagigen Schicht von Graphen gestaltet sich dies schwierig. Die Leitf\"ahigkeit sinkt niemals unter einen bestimmten Wert von $e^2/h$ und es ist nicht m\"oglich, die Ladungstr\"ager in Graphen einzuschr\"anken \cite{Katsnelson2006}. F\"ur dieses Problem verspricht doppellagiges Graphen (BLG) Abbhilfe. Wenn BLG einem elektrostatischen Feld ausgesetzt wird, haben die beiden Schichten jeweils ein leicht unterschiedliches Potential. Das f\"uhrt dazu, dass sich im BLG-Spektrum eine Bandl\"ucke \"offnen l\"asst, abhh\"angig von der St\"arke des elektrischen Feldes. Das legt die Vermutung nahe, dass bei Transportmessungen mit BLG ein isolierender Zustand erreicht werden k\"onnte.

Transportmessung an Supraleiter-BLG-Supraleiter-Schaltungen -- eine Schaltung bestehend einer BLG-Fl\"ache, die an den Seiten mit zwei Supraleitern kontaktiert ist -- finden nicht den erwarteten isolierenden Zustand \cite{Zhu2017}. Durch Messung des kritischen Stroms $I_c$ in Abh\"angigkeit des angelegten Magnetfeldes $B$ konnte R\"uckschluss auf die Verteilung der Stromdichte innerhalb der Probe getroffen werden. Anhand der Ergebnisse der Stromdichteverteilung wurde die Vermutung aufgestellt, dass Stromtransport durch Kan\"ale an den Kanten der Probe f\"ur die endliche Leitf\"ahigkeit der Probe verantwortlich seien. 

Am Institut f\"ur Nanotechnologie in der Arbeitsgruppe von Ralph Kruppke werden Supraleiter-BLG-Supraleiter-Schaltungen mit einem sogenannten \emph{weak link}, also einer Engstelle in der Probe, durch die der Strom passieren muss, untersucht. Die Experimentelle Untersuchung dieses \emph{weak links}, der die Form eines Quanten-Punkt-Kontaktes hat, wird von einer analytischen Untersuchung des Kontakts im Rahmen einer quasiklassischen Transporttheorie und einer numerischen Untersuchung mittels Tight-binding Methode gest\"utzt. Die Ergebnisse dieser Untersuchung sind in \cite{Kraft2017} ver\"offentlicht. 


\subsection*{Gliederung der Arbeit} 

Die vorliegende Arbeit besch\"aftigt sich mit der Physik solcher Supraleiter-Normalleiter-Supraleiter-Schaltungen (SNS-Schaltungen). Der Suprastrom f\"ur den QPC-Aufbau wird im Kontext dieser quasiklassischen Transporttheorie untersucht. Die Ergebnisse der Untersuchung werden durch die Berechnung der Suprastr\"ome mittels Tight-Binding-Methode erg\"anzt.

In Kapiteln \ref{ch:basics} und  \ref{ch:basics-numerical} werden die Grundlagen aufbereitet, die zum Verst\"andnis dieser Arbeit notwendig sind. In Kapitel \ref{ch:basics} werden Prozesse n\"aher betrachtet, die an der Grenzfl\"ache von Supraleitern und Normalleitern relevant sind. Zu nennen ist hier insbesondere die Andreev Reflektion. Darauf aufbauend wird erl\"autert, wie es in einer SNS-Schaltungen zu Stromfluss kommen kann. Kapitel  \ref{ch:basics-numerical} geht auf Details der Tight-Binding-Methode ein und erl\"autert den Streumatrix-Ansatz. Zun\"achst wird der Tight-Binding-Hamiltonian f\"ur Graphen und BLG hergeleitet und darauf aufbauend der Modell-Hamiltonian f\"ur das QPC-System erkl\"art. Die experimentellen Fragestellungen, die f\"ur diese Arbeit relevant sind, werden in Kapitel \ref{ch:experiment} eingef\"uhrt. Die experimentellen Befunde der SNS-Schaltung mit einem QPC-Aufbau werden vorgestellt und diskutiert.

Die Ergebnisse, die f\"ur den Suprastrom im Rahmen der quasiklassischen Transporttheorie folgen, werden in Kapitel  \ref{ch:analyticalmodel} vorgestellt. Ein zentrales Ergebnis ist, dass der \"Ubergang von einem oszillierendem zu einem Gauss-f\"ormigen Muster in \"Ubereinstimmung mit den experimentellen Befunden gefunden wird. Weiterhin wird der Stromtransport entlang der Kanten der Probe im Rahmen dieser Transporttheorie untersucht: Die Berechnungen zeigen einerseits, wie der kritische Strom von Randkan\"alen beeinflusst wird und andererseits, dass in diesem speziellen Aufbau Transport entlang dieser Kan\"ale unwahrscheinlich ist.

In Kapitel \ref{ch:numerical-results} wird neben dem QPC-Aufbau noch ein weiterer, Wellenleiter-f\"ormiger Aufbau mit Tight-Binding-Modellen untersucht. Die berechneten Ergebnisse f\"ur Suprastrom und Leitf\"ahigkeit werden pr\"asentiert und es wird auf den Einfluss von Unordnung in der Probe und Defekte an den Kanten eingegangen. Die berechneten Ergebnisse f\"ur den QPC-Aufbau decken sich gut mit den experimentellen Befunden.


%%%%%%%%%%%%%%%%%%%%%%%%%%%%
%	Ergebnisse?        %
%%%%%%%%%%%%%%%%%%%%%%%%%%%%
\subsection*{Ergebnisse}
Das zentrale Ergebnis dieser Arbeit ist, dass
\begin{equation}
J(\tilde{\chi}(y_1, y_2), \phi) \propto \int \int_{-W/2}^{~W/2} dy_1 dy_2 \frac{ \mathcal{J}(\tilde{\chi}(y_1, y_2)) }{ \left( 1 - \left(\frac{y_1 - y_2}{L}\right)^2 \right)^2 }
\end{equation}
Die Potenz im Nenner f\"ur den Strom durch die SNS junction geht mit der Potent von 2. Dies widerspricht dem Ergebnis von \cite{Barzykin1999}. 
Die Stromdichte f\"ur das QPC wurde aufgestellt, sie lautet
\begin{equation}
J(\tilde{\chi}(y_1, y_2), \phi) \propto \int \int_{-W/2}^{W/2} d y_1 d y_2 \frac{\cos \left( \frac{\pi \phi}{W}(y_1 + y_2) \right)}{\left[ 1 + \left(\frac{y_1 - y_2}{L}\right)^2\right]^2} \label{eq:josephson_current},
\end{equation}
Der kritische Strom wurde ausgewertet f\"ur den Grenzfall $\phi \rightarrow 0$, in dem eine parabolische Funktion abh\"angig von W/L f\"ur das QPC gefunden wurde. In dem anderen Grenzfall, $\phi \rightarrow \infty$, wird der exponentielle Abfall f\"ur gro{\ss}e $\phi$ korrekt wiedergegeben. Es wurde auch das Verhalten der Randstr\"ome im Rahmen der quasi-klassischen N\"aherung untersucht und es zeigt sich, dass f\"ur verschiedenen Tranmissionskoeffizienten $\mathcal{T}_e$ der Kante und $\mathcal{T}_q$ des QPC, der kritische Strom durch einen Faktor 
\begin{equation}
\mathcal{C} =  \frac{| \mathcal{T}_q / \mathcal{T}_e + \cos \left( \pi \phi \right)/2 |}{\mathcal{T}_q / \mathcal{T}_e + 1/2}
\end{equation}
moduliert wird. Diese Modulation zeigt sich schon ab einem Wert von $\mathcal{T}_e/ \mathcal{T}_q = 1 / 100$ und ist ein starker Hinweis daraus, dass in den beobachteten Daten keine Kantenstr\"ome zu finden sind. 

\section{Ausblick}
%%%%%%%%%%%%%%%%%%%%%%%%%%%%
%      Ausblick		   %
%%%%%%%%%%%%%%%%%%%%%%%%%%%%

Die im Ramen dieser Arbeit vorgestellte Transporttheorie kann in verschiedenen Aspekten erweitert werden. Es wurde demonstriert, dass diese Theorie einen \emph{weak link} in Form eines QPC gut modellieren kann. Es ist deshalb denkbar, \"ahnliche Formen von weak links zu modellieren.

In dieser Arbeit wurde ein kurzer SNS-Aufbau betrachtet. Das hat bestimmte N\"aherungen als Konsequenz, die die Rechnungen vereinfachen. F\"ur den anderen Grenzfall einer langen SNS-Schaltung, ist zum einen Streuung an den Kanten ein wichtiger Bestandteil zum Stromtransport und muss ber\"ucksichtigt werden. Zum anderen muss dann die Stromdichte modifiziert werden, die einfachste From der Joesphson Gleichung ist dann nicht mehr ausreichend.


%Die quasi-klassische Transporttheorie kann f\"ur kompliziertere F\"alle erweitert werden. 
%Analytische Rechnungen k\"onnen f\"ur weitere Setups berechnet werden, f\"ur kompliziertere Setups. Dabei kann beispielsweise Mehrfachstreuung und Anisotrope Streeuung hinzugezogen werden. 

%Im Kontext auf die Experimente sind die M\"oglichkeiten beinahe unbeschr\"ankt. Es gibt Ans\"atze, f\"ur Spin Orbit Coupling in BLG, die nutzen und afw\"andige experimentelle Setups beschreiben.



