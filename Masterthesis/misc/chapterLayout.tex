%%%%%%%%%%%%%%%%%%%%%%%%%%%%%%%%%%%%%%%%%%%%%%%%%%%%%%%%%%%%
%%%%%%%%%%%%%%%%%%%%%%%%%%%%%%%%%%%%%%%%%%%%%%%%%%%%%%%%%%%%
%%%%%%%                            %%%%%%%%%%%%%%%%%%%%%%%%%
%%%%%   chapter headings             %%%%%%%%%%%%%%%%%%%%%%%
%%%%%%                             %%%%%%%%%%%%%%%%%%%%%%%%%
%%%%%%%%%%%%%%%%%%%%%%%%%%%%%%%%%%%%%%%%%%%%%%%%%%%%%%%%%%%%
%%%%%%%%%%%%%%%%%%%%%%%%%%%%%%%%%%%%%%%%%%%%%%%%%%%%%%%%%%%%

%%% INFO: when using different fonts, the chapter headings may look
%%% different and the given sizes do no longer apply!

\colorlet{chapter}{darkblue}
\addtokomafont{chapter}{\color{black}}

\makeatletter% siehe De-TeX-FAQ
\renewcommand*{\chapterformat}{%
\begingroup% damit \unitlength-Änderung lokal bleibt
\setlength{\unitlength}{1mm}%
\begin{picture}(20,40)(0,5)%
\setlength{\fboxsep}{0pt}%
%\put(0,0){\framebox(20,30){}}%
%\put(0,20){\makebox(20,20){\rule{20\unitlength}{20\unitlength}}}%
\put(20,15){\color{black}\linethickness{0.4pt}\line(1,0){\dimexpr\textwidth-20\unitlength\relax\@gobble}}% line
\put(0,0){\makebox(18,20)[r]{\fontsize{20\unitlength}{1}\color{black}\selectfont\thechapter %\kern-.04em% Ziffer in der Zeichenzelle nach rechts schieben
}}%
\put(20,15){\makebox(\dimexpr \textwidth-20\unitlength\relax\@gobble,\ht\strutbox\@gobble)[l]{%\ %\normalsize\color{black}\chapapp~\thechapter\autodot 
}}%
\end{picture} % <- Leerzeichen ist hier beabsichtigt!
\endgroup
}%
\makeatother
